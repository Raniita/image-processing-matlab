\documentclass[12pt]{article}
\usepackage[spanish, english, es-tabla]{babel}
\usepackage[utf8]{inputenc}
\usepackage[left = 2cm, right = 2cm, bottom = 2cm, top = 3cm]{geometry}
\usepackage{amsmath, amssymb}
\usepackage{graphicx}
\usepackage{hyperref}
\usepackage{listings}
\usepackage{courier}

\usepackage[dvipsnames]{xcolor}

\renewcommand{\lstlistingname}{Código}
\renewcommand{\lstlistlistingname}{Listado de códigos}

% https://tex.stackexchange.com/questions/60209/how-to-add-an-extra-level-of-sections-with-headings-below-subsubsection
\newcommand{\subsubsubsection}[1]{\paragraph{#1}\mbox{}\\}
\setcounter{secnumdepth}{4}
\setcounter{tocdepth}{4}

% Configure lstlisting
\definecolor{codegreen}{rgb}{0,0.6,0}
\definecolor{codegray}{rgb}{0.5,0.5,0.5}
\definecolor{codepurple}{rgb}{0.58,0,0.82}
\definecolor{backcolour}{rgb}{0.95,0.95,0.92}

\lstdefinestyle{mystyle}{
	backgroundcolor=\color{backcolour},   
	commentstyle=\color{codegreen},
	keywordstyle=\color{magenta},
	numberstyle=\tiny\color{codegray},
	stringstyle=\color{codepurple},
	basicstyle=\ttfamily\footnotesize,
	breakatwhitespace=false,         
	breaklines=true,                 
	captionpos=t,                    
	keepspaces=true,                 
	numbers=left,                    
	numbersep=5pt,                  
	showspaces=false,                
	showstringspaces=false,
	showtabs=false,                  
	tabsize=2
}

\lstset{style=mystyle}


\begin{document}
	\selectlanguage{spanish}
	
	\title{Proyecto 1. Procesado de imágenes con \texttt{MATLAB} \\ \textit{\textbf{\large Máster Universitario en Ingeniería de Telecomunicación}} \\ \textit{\large Procesado de señales acústicas e imágenes}}
	\author{Enrique Fernández Sánchez}
	
	\maketitle
	
	\tableofcontents
	
	\addcontentsline{toc}{section}{Listado de códigos}
	\lstlistoflistings
	
	\pagebreak
	
	\section{Anonimizado}
	
	\begin{lstlisting}[language=Matlab, caption={Implementación en MATLAB para anonimizado}]
% 1 - Anonimizado Enrique 2021/2022
% Ref: https://es.mathworks.com/help/images/ref/imcrop.html
clear;

% Cargamos la imagen
img = imread('anonimizado.jpg');

% Representamos la imagen original
figure
imshow(img);
title('Imagen inicial (sin recortar)')

% Recortamos la imagen con un rectangulo
[crop_img, rect_crop] = imcrop(img);
%rect_crop = [120 300 2750 1400];    % [xmin ymin width height]
%crop_img = imcrop(img, rect_crop);

% Comparativa imagen original vs imagen recortada
figure
%suptitle('Imagen Anonimizada')

subplot(1,2,1)
imshow(img)
title('Imagen inicial')

subplot(1,2,2)
imshow(crop_img)
title('Imagen recortada utilizando un rectangulo')
	\end{lstlisting}
	
	\section{Contraste}
	
	\section{Iluminación}
	
	\section{Suavizado}
	
	\section{Realzado}
	
	\section{Ruido}
	
	\section{Patrones}
	
	\section{Pseudocoloración}
\end{document}